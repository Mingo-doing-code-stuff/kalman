\documentclass{beamer}

\usepackage[utf8]{inputenc}
\usepackage{amsmath}
\usepackage{graphicx}
\usepackage{xcolor}
\usepackage{style}
\usepackage[ddmmyyyy]{datetime}

\title{Kalman Filter}
\subtitle{Grundlagen}
\author{P.Schön, C.Thein}
\date{\today}
\renewcommand{\dateseparator}{.}



\begin{document}

\frame{\titlepage}

\begin{frame}
    \frametitle{Inhaltsverzeichnis}
    \tableofcontents
\end{frame}

\section{Einleitung}

\begin{frame}
    \frametitle{Was ist das Kalman Filter?}
    Das Kalman Filter ist ein mathematisches Verfahren zur iterativen Schätzung von Parametern zur
    Beschreibung von Systemzuständen.

    Dabei wird wiederholt eine Vorhersage über einen Parameterwert
    abgegeben, mit dem fehleranfälligen Messwert kombiniert, und erneut gnutzt um daraus eine Vorhersage
    zu treffen.
\end{frame}

\section{Vereinfachte Erklärung}




\begin{frame}
    \frametitle{Ablauf des Kalman Filters}

    \textbf{Vorhersage}

    1. Denn nächten Zustand darstellen: \( \hat{x}_{k} = A\hat{x}_{k-1}+Bu_{k-1} \)

    2. Die Fehlerkovarianz vorausberechnen: \( P_{k}=AP_{k-1}A^{T}+Q \)

    \textbf{Korrektur}

    3. Den Kalman Gain berechnen: \( K_{k}=P_{k}H^{T}(HP_{k}H^T+R)^{-1} \)

    4. Die Schätzung mit \(z_k\) aktualisieren: \( \hat{x}_{k}=\hat{x}_{k}+K_{k}(z_{k}-H\hat{x}_{k}) \)

    5. Die Fehlerkovarianz aktualisieren: \( P_{k}=(I-K_{k}H)P_{k} \)
\end{frame}


\section{EXAMPLES}

\begin{frame}
    \frametitle{Ablauf des Kalman Filters}
    \begin{itemize}
        \item \textbf{Fettgedruckt}
        \item \textit{Kursiv}
        \item \underline{Unterstrichen}
        \item \texttt{Monospaced}
    \end{itemize}
\end{frame}

\begin{frame}
    \frametitle{Aufzählungen}
    \begin{itemize}
        \item Erster Punkt
        \item Zweiter Punkt
        \item Dritter Punkt
    \end{itemize}
\end{frame}

\begin{frame}
    \frametitle{Mathematische Ausdrücke}
    \begin{itemize}
        \item Inline: \(E = mc^2\)
        \item Displayed:
              \[
                  \int_0^\infty e^{-x^2} \, dx = \frac{\sqrt{\pi}}{2}
              \]
    \end{itemize}
\end{frame}

\begin{frame}
    \frametitle{Bilder einfügen}
    \begin{figure}
        \centering
        \includegraphics[width=0.5\textwidth]{sample}
        \caption{Ein Beispielbild}
    \end{figure}
\end{frame}

\section{Zusammenfassung}

\begin{frame}
    \frametitle{Zusammenfassung}
    In dieser Präsentation haben wir die grundlegenden Elemente von LaTeX vorgestellt, darunter:
    \begin{itemize}
        \item Textformatierung
        \item Aufzählungen und Listen
        \item Mathematische Ausdrücke
        \item Bilder einfügen
    \end{itemize}
\end{frame}

\end{document}
\ProvidesPackage{beamerthemesss}
\mode<presentation>
\usepackage{xcolor}
\usepackage{tikz}
\RequirePackage{enumitem}


\definecolor{orange_100}{RGB}{255, 107, 0} 	% orange_100
\definecolor{orange_80}{RGB}{255, 137, 51} 	% orange_80
\definecolor{black_100}{RGB}{0, 0, 0} 		% black_100
\definecolor{black_80}{RGB}{50, 50, 50} 		% black_80
\definecolor{black_60}{RGB}{100, 100, 100} 	% black_60
\definecolor{white}{RGB}{255, 255, 255} 		% white


% Define a command to set the foreground color of lists
\newcommand{\setlistcolor}[1]{
    \setlist[enumerate]{label=\textcolor{#1}{\textbullet}}
    \setlist[itemize]{label=\textcolor{#1}{\textbullet}}
}
\setlistcolor{orange_80}

% Schriftarten einstellen
\setbeamerfont{title}{size=\huge,series=\bfseries}
\setbeamerfont{subtitle}{size=\small,series=\bfseries}
\setbeamerfont{frametitle}{size=\large,series=\bfseries}
\setbeamerfont{normal text}{size=\small}

% Farben anwenden
\setbeamercolor{title}{fg=orange_100}
\setbeamercolor{subtitle}{fg=black_60}
\setbeamercolor{frametitle}{fg=orange_100}
\setbeamercolor{section in toc}{fg=black_80}
\setbeamercolor{normal text}{fg=black_80}
\renewcommand{\figurename}{\textcolor{orange_100}{Fig.}}

% Layout anpassen
\setbeamertemplate{navigation symbols}{}

% Customize header and footer with dotted lines made of circles
\setbeamertemplate{headline}{
    \begin{tikzpicture}[remember picture,overlay]
        \ifnum\theframenumber=1
            \node[anchor=north east, inner sep=0pt, yshift=-0.5mm, xshift=-2mm] at (current page.north east) {
                \includegraphics[height=4.8mm]{logo.png}
            };
        \fi
        \draw[thick, dash pattern=on 0pt off 2\pgflinewidth, line cap=round, color=orange_100]
        ([yshift=-7.5mm]current page.north west) --
        ([yshift=-7.5mm]current page.north east);
    \end{tikzpicture}
}

\setbeamertemplate{footline}{
    \begin{tikzpicture}[remember picture,overlay]
        \ifnum\theframenumber>1
            \draw[thick, dash pattern=on 0pt off 2\pgflinewidth, line cap=round, color=orange_100]
            ([yshift=8mm]current page.south west) --
            ([yshift=8mm]current page.south east);
            \node[anchor=east, inner sep=0pt, yshift=2.5mm, xshift=-5mm] at (current page.south east) {
                \color{black_60}
                \insertframenumber/\inserttotalframenumber
            };
            \node[anchor=west, inner sep=0pt, yshift=3.5mm, xshift=-2mm] at (current page.south west) {
                \includegraphics[height=7.8mm]{watermark.png}
            };
        \fi
        \ifnum\theframenumber=1
            \node[anchor=west, inner sep=0pt, yshift=6mm, xshift=-4mm] at (current page.south west) {
                \includegraphics[height=9.8mm]{watermark.png}
            };
        \fi

    \end{tikzpicture}
}

% Abschnittsüberschriften
\setbeamercolor{section in head/foot}{bg=orange_100,fg=white}
\setbeamerfont{section in head/foot}{series=\bfseries}
\setbeamertemplate{section in head/foot shaded}{\color{orange_100!50}}

\setbeamertemplate{frametitle}{
    \begin{tikzpicture}[remember picture,overlay]
        \node[
            anchor=base west,
            fill=white,
            yshift=-3.9mm,
            xshift=-2mm
        ] {\bfseries\insertframetitle};
    \end{tikzpicture}
}


\endinput

